\documentclass[a4paper,twocolumn]{article}

\usepackage[margin=2.5cm]{geometry}
\usepackage{xcolor}
\usepackage{listings}
\lstset{%
frame=single,
% framesep=13pt,
% framexleftmargin=7pt,
xleftmargin=20pt,
% fillcolor=\color{gray!10},
backgroundcolor = \color{gray!5},
rulecolor=\color{black!30},
numberstyle=\sffamily\color{black!75},
numbers = left,
% breaklines = true,
% showspaces = false,
basicstyle = \ttfamily\footnotesize,
tabsize = 2,
captionpos=b,
language=Haskell,
morekeywords={data,where},
keywordsprefix=\#,
alsoletter=\#
}

\newcommand{\malfunction}{\texttt{malfunction}}

\title{An Idris Foreign Function Interface to OCaml}
\author{Robert Atkey  $\langle$\texttt{robert.atkey@strath.ac.uk}$\rangle$ \\
  Ioan Luca $\langle$\texttt{FIXME}$\rangle$}
\date{}

\begin{document}

\maketitle

\section{Introduction}



\begin{itemize}
\item Idris is a dependently typed programming langauge
\item OCaml isn't, but it has a lot of libraries
\item Dolan's \texttt{malfunction} backend enables compilation from
  Idris to machine code via the OCaml compiler, but only provides
  limited support for calling external OCaml functions.
\item We would like to exploit the possibility of compiling via the
  OCaml backend to not only generate more efficient programs, but also
  to access OCaml's library ecosystem.
\item This entails constructing an FFI for Idris to interface with
  OCaml programs. FFIs have been constructed for other backends, but
  these backends are either to languages with simpler type systems
  (e.g., C), or to languages with no static type system at all
  (JavaScript, Erlang). When both sides of the FFI have complex type
  systems, a common solution is to restrict to a subset of the
  intersection of the two type systems, often only allowing
  monomorphic, non higher order types. We want to explore doing
  something more expressive, so that we don't end up writing
  translation code in OCaml for Idris to interface with.
\end{itemize}

What we've done:
\begin{enumerate}
\item Improved the Idris \malfunction{} backend by not using Idris's
  defunctionalised representation, and by using the same compilation
  scheme
\item Made an FFI definition for Idris that allows basic interaction
  between OCaml and Idris code. This has allowed us to write a simple
  web server in Idris, using the Cohttp library.
\end{enumerate}

\section{Background}

\paragraph{Idris}

Idris \cite{idris} is a dependently typed language with a strict
evaluation semantics. Idris's implementation has been written to allow
pluggable backends. C and JavaScript backends are included with the
main distribution, and

\paragraph{Malfunction}



\section{OCaml Types in Idris}

\begin{lstlisting}
data OCaml_Types : Type -> Type where
   OCaml_Str   : OCaml_Types String
   OCaml_Float : OCaml_Types Double
   OCaml_Bool  : OCaml_Types Bool
\end{lstlisting}

\paragraph{Primitive types}



\paragraph{Function Types}

\paragraph{Algebraic datatypes}

\paragraph{Records}

\paragraph{Modules}

\section{Future Work}

\begin{itemize}
\item Polymorphism, properly. This is tricky because we need to insert
  the right coercions between Idris and OCaml types (especially the
  effectful ones). Do we need runtime type representations?
\item Algebraic datatypes, generally. We can write an encoding of
  algebraic datatypes in Idris, but this will be separate from Idris's
  native algebraic datatypes.
\item Properly reflecting the module system (related to the
  polymorphism problem)
\item Generating Idris type definitions from \texttt{.cmi} files.
\end{itemize}

\end{document}
