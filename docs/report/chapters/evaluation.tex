%15
\section{The value of my work}
OCaml developers can also benefit from this backend by using the 
software libraries they are already familiar but in a more expressive 
type system.

The most exciting points of my work are the fact you can use a library like
OCaml graphics from Idris and even build a unikernel in Idris using Mirage as long as you provide the configuration file manually and the interface file (.mli).


\section{Were my requirements met?}

Although I didn't manage to implement and document of the requirements
listed in the 3rd chapter.

I think I still managed to achieve most of them. my work explores some interesting ideas about interfacing with
OCaml. 

\section{My own performance}
I underestimated the project
which was quite involved; looking at some of the other back-ends (mostly
toy backends)
I thought this project was going to be much easier.

The OCaml expressiveness comes at a cost which is complexity and
I believe I underestimated this complexity. 
I was familiar to Haskell to an extent. 

% \subsection{Modules}
% unikernel run
% make them user friend

% Functors in a way try to fulfil some of the things that
% dependent types do


% the Idris type system should be powerful enough to express them

% Added flag to Malfunction
% Improve the previous back-end  (ergonomic problems)
% - added AST allows to now generate AST instead, potentially reusable
% - solve some laziness bugs with the help of Malfunction
% - faster code generation using Data.Text
% - implement more primitives
% - modularize the code base
% - added support for flags from the outside --cg-opt
% - can export functions as
% - a few more test cases

% on the side of idris, an IO for interfacing with OCaml library

% an OCaml wrapper for interfacing with the Obj OCaml library for
% interacting with OCaml runtime values.

% a few examples of Idris interfacing with OCaml like the
% grahphics

% a Docker image suitable for developing haskell and OCaml
% based projects, Opam + stack etc.

% a basic simple encoding of OCaml complex module system in
% a dependent typed way using Idris upon which to build
% upon and support properly
