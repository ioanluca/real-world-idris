\section{Towards a more practical Idris}

Idris\footnote{\url{https://www.idris-lang.org/}} is a pure functional,
general purpose, dependently typed programming language.
Idris programs use types as the main building block which allows it
to encode complex invariants directly into source code.
This means that the compiler checks programs for
correctness before they run, enabling
``lightweight verification''\footnote{specification is
	defined early in the development phase as a language construct;
	the implementing code is then guaranteed to work against it}.
This has the potential to drastically reduce the costs
of reliable software by integrating verification into development
instead of considering it a separate concern.

Despite being novel and popular amongst Programming Language research
groups, Idris did not catch the attention of industrial developer
communities.
This is partly because the language lacks a high performance
runtime system and partly because its library ecosystem has poor
support for fast and reliable software.

%fixme what's with this title
\subsubsection{The Idris-Malfunction code generation backend}

Idris-Malfunction is a recent experimental compiler backend targeted at
solving the performance gap between Idris and other compiled languages
by compiling Idris to the OCaml ``battle-tested'' runtime system
via a small but expressive intermediate language
called
Malfunction\footnote{\url{https://github.com/stedolan/malfunction}}.

This backend was developed, under the supervision of
Dr.~Robert Atkey, as part of a research internship
within the Mathematically Structured Programming (MSP) group at the
University of Strathclyde.

``Real World Idris'' aims to upgrade Idris-Malfunction
with a Foreign Function Interface (FFI) to OCaml.
The idea is to enable the two-way interoperability between these two
languages.

This would essentially allow Idris programs to access the mature OCaml
library ecosystem, which would ultimately make Idris more suitable
for practical programming.


% todo what write here?
\section{Objectives}
The project involved mapping the OCaml types to Idris equivalents,
implementing a typed API to the OCaml runtime and there were also
opportunities to fix some of the ergonomic issues within the
Idris-Malfunction.

The upgrade will be carried out using the Haskell programming language which is
what the compiler back-end is currently written in.
It will, however, involve coding in OCaml and Idris as well.

Upon completion of a working FFI, a set of Idris programs that
call OCaml code will be presented as proof of concept.


\section{Contributions}
The main contributions of this project are the enhancements brought
to the Idris-Malfunction code generation backend and a
system that enables the interoperability between Idris and OCaml
to an extent, which is documented and supported by running code
examples. In addition there is some basic support for building
Haskell and OCaml based applications inside
Docker\footnote{\url{https://en.wikipedia.org/wiki/Docker_(software)}}
containers.


\section{Outline}
%todo come back to write this
Chapter 8 concludes this dissertation/report.