\section{Towards a more practical Idris}

Idris\footnote{\url{https://www.idris-lang.org/}} is a pure functional,
general purpose, dependently typed programming language.
Idris programs use types as the main building block which allows it
to encode complex invariants directly into source code.
This means that the compiler checks programs for
correctness before they run, enabling
``lightweight verification''\footnote{specification is
	defined early in the development phase as a language construct;
	the implementing code is then guaranteed to work against it}.
This has the potential to drastically reduce the costs
of reliable software by integrating verification into development
instead of considering it a separate concern.

Despite being novel and popular amongst Programming Language research
groups, Idris did not catch the attention of industrial developer
communities.
This is partly because the language lacks a high performance
runtime system and partly because its library ecosystem has poor
support for fast and reliable software.

\subsubsection{The Idris-Malfunction code generation backend}

Idris-Malfunction is a recent experimental compiler backend targeted at
solving the performance gap between Idris and other compiled languages
by compiling Idris to the OCaml ``battle-tested'' runtime system
via a small but expressive intermediate language
called
Malfunction\footnote{\url{https://github.com/stedolan/malfunction}}.

This backend was developed by myself, under the
supervision of Dr.~Robert Atkey, as part of a research internship
within the Mathematically Structured Programming (MSP) group at the
University of Strathclyde.

``Real World Idris'' aims to upgrade Idris-Malfunction
with a Foreign Function Interface (FFI) to OCaml.
The idea is to enable the two-way interoperability between these two
languages.

This would essentially allow Idris programs to access the mature OCaml
library ecosystem, which would ultimately make Idris more suitable
for practical programming.

\section{Objectives}
This project has two main goals, as detailed below:

\subsubsection{Fix some ergonomic problems with Idris-Malfunction}
\begin{enumerate}
	\item Identify and fix \emph{bottlenecks} to increase code generation
	      speed (\textbf{basic}, executable)
	\item Add support for more Idris
	      \emph{primitives} (\textbf{basic}, executable)
	\item Add support for \emph{floating-point} numbers
	      (\textbf{basic}, executable)
	\item Add \emph{native} support for \emph{lazy evaluation}
	      (\textbf{intermediate}, executable)
	\item \emph{Refactor} the code base, improving the architecture and
	      the compilation time (\textbf{intermediate}, source)
	\item Add support for Unicode (\textbf{advanced}, executable)
\end{enumerate}

\subsubsection{Create the FFI OCaml}
\begin{enumerate}
	\item Map primitive Idris types to OCaml equivalents
	      (\textbf{basic}, source and document)
	\item Call Idris basic functions
	      back from OCaml (\textbf{intermediate}, source)
	\item Provide a typed API to the OCaml runtime system
	      (\textbf{intermediate}, Idris library and document)
	\item Add option to compile Idris source as OCaml bytecode
	      objects to later be linked in OCaml libraries
	      (\textbf{intermediate}, executable)
	\item Map algebraic data types to OCaml equivalents
	      being able to exchange complex data
	      (\textbf{advanced}, source and document)
	\item Export Idris functions and data types to
	      safely exchange complex data between the two languages
	      (\textbf{advanced}, executable)
	\item Describe the OCaml module system with dependent types
	      (\textbf{advanced}, executable and document)
	\item Explore a potential translation of Idris dependent types
	      to OCaml (\textbf{advanced}, document)
\end{enumerate}

\subsubsection{Prove the concept with a ``real world'' use case}
\begin{enumerate}
	\item Create a small set of Idris programs that call simple
	      OCaml functions (\textbf{basic}, source)
	\item Create an Idris program that uses OCaml \emph{Graphics}
	      library \\ (\textbf{intermediate}, executable)
	\item Create an Idris program that uses MirageOS\footnote{\url{
			      https://mirage.io/}} to construct a simple
	      unikernel (\textbf{advanced}, executable)
	\item Create an Idris program that uses MirageOS to construct an
	      HTTP server as a unikernel (\textbf{advanced}, executable)
\end{enumerate}



\section{Contributions}
The main contributions of this project are the enhancements brought
to the Idris-Malfunction code generation backend and a
system that enables the interoperability between Idris and OCaml
to an extent, which is documented and supported by running code
examples. In addition there is some basic support for building
Haskell and OCaml based applications inside
Docker\footnote{\url{https://en.wikipedia.org/wiki/Docker_(software)}}
containers.


\section{Outline}
%todo come back to write this
Chapter 8 concludes this dissertation/report.