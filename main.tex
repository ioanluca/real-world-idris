\documentclass[a4paper]{report}

	\usepackage[utf8]{inputenc} % TODO
	\usepackage{indentfirst} % FIXME matter of taste
	
\begin{document}


\title{Real World Idris \\ Project Scope and Outline}
\author{Ioan Luca}
\date{\today}
\maketitle

\pagenumbering{roman}

% \tableofcontents
% \listoffigures
% \listoftables

% \chapter*{Acknowledgements}

% \begin{abstract}
% \end{abstract}

\pagenumbering{arabic}

\chapter{Introduction}
Submission of this, along with feedback from your supervisor, must be done
via Myplace.
This deliverable is timed to ensure that you get together with your
supervisor without undue delay and scope out the project.
The submission should contain the following.
You should write a short report (no more than 4 pages) that includes:
\begin{itemize}
	\item An overview of what the project is to achieve. This should be
	      reasonably
	      substantial
	      – either two or three paragraphs in length.
	\item A preliminary list of achievable objectives for the project.
	\item A preliminary survey of related work.
	\item A brief overview of the methodology to be followed
	      (i.e., how the specification, design, implementation and
	      verification
	      of the project will be approached).
	\item A brief overview of how the end product will be evaluated.
	\item An initial project plan scheduling the development of the project
	      from now until the
	      project submission deadline.
	\item A choice of marking scheme (see the section Marking Schemes ) to
	      be used to assess the
	      project, together with the reasons for choosing that particular
	      scheme
\end{itemize}
At the end of the report you should include comments made by your supervisor
about your report (no more than two paragraphs) - these paragraphs should
comment on the extent to which you appear to understand the nature and scope of the project, and to appreciate where the challenges in it lie.
It is expected that the objectives of the project will be refined as the project
progresses and that the related work will be expanded and appropriately described
for the final report. It is also expected that the project plan will change as the
project progresses.


\section{Overview}

\subsection{Background}
''Real World Idris'' aims to extend the Idris ecosystem with a Foreign
Function Interface (FFI) to OCaml, to be used as a mechanism for calling
functions and services written in OCaml from Idris.

The idea of this upgrade is to enable Idris programs to build on top of
''battle-tested'' software which would ultimately make Idris more suitable
for practical programming. For instance, one could write a secure,
high-performance network application in Idris running as a unikernel
constructed by MirageOS, an OCaml built library operating system.

\subsection{Idris}
% intro
Idris is a general purpose, dependently typed programming language that has an
advanced type system which encodes complex properties about programs into
types, which means that programs are checked for correctness before they run.
(TyDe)
% TODO talk about TyDe
Idris is at the forefront of a new generation of programming languages
that support “lightweight verification”, reducing the costs of reliable
software by integrating verification into development, instead of considering
it a separate concern.

% Idris is new and slow, maybe hard?
Despite being novel and popular in the Programming Language research community
Idris did not catch the attention of the industrial developer communities.
This resulted in a lack of reliable, secure and optimized tools and libraries
in the Idris ecosystem and I am going to highlight some of the reasons for
why that happened:
%TODO Peformance
Firstly, despite having an extensible compiler that targets more than one
platform, the Idris main/native runtime is currently not performant enough
compared to its industry counterparts.
Some work has been put into that bla bla
%TODO talk about idris-malfunction

% Talk more about the industry adopting a new language? maybe research
% TODO how intuitive and easy to learn is it
Secondly, Idris has a steep learning curve partly because it requires
a somewhat strong background in functional programming on which it relies
heavily.
In addition, the ''Type Driven'' paradigm is almost entirely based on the
Dependent Types theory which is widely inaccessible outside the academic world.



\subsection{OCaml}
% intro
OCaml is a mature language which benefits from advanced and
reliable industry software being written
in it.

% OCaml on the other hand is mature and popular which allowed for battle-tested 
% libraries like Mirage to be developed. 



\subsection{idris-malfunction}
% TODO about the work this summer why lambdas what made it faster etc.
There are currently multiple code generation backends for Idris and amongst
them, there is one which compiles Idris to OCaml bytecode, which is then
turned into native executable code by the optimized OCaml compiler.
This backend was developed by myself, being supervised by Dr. Robert Atkey
as part of a research internship within the Mathematically Structured
Programming (MSP) group at the University of Strathclyde.
%The internship was funded by  EPSRC as part of their Vacation Bursary scheme.

% The project would be beneficial for expanding my knowledge of programming
% language design and implementation. It will also require me to improve my skills
% in 3 different languages. There will
% also be opportunities to potentially learn about specialized kernels, some low-level
% networking and deployment when tackling the advanced deliverable.

\subsection{Mirage}
% Unikernels
% Secure
% Fast

\subsection{Worked involved}
The work will involve mapping the OCaml types to Idris equivalents,
implementing the FFI and there will also be opportunities to fix some of the
ergonomic issues within the current Idris-OCaml backend.
The upgrade will be carried out using the Haskell programming language which is
what the compiler backend is currently written in. The project will involve
coding in OCaml and Idris as well.




\section{Objectives}
Basic –  Foreign Function Interface (FFI) from Idris to Ocaml; Specifically,
being able to call
OCaml functions from Idris

Intermediate – Being able to safely exchange data between Idris and OCaml;
Specifically,
functionality that maps between the OCaml type system and the Idris type system
or vice versa when
one of the following happens:
- Idris typed arguments are passed into OCaml functions
- OCaml typed values are turned from OCaml functions

Advanced – Implement a useful piece of software in Idris and validate
it in a ''real-world'' scenario;
For example, write an HTTP server in Idris and deploy it as a MirageOS
unikernel on
Amazon Web Services.

\section{Preliminary survey of related work}

\subsection{Requirements Gathering}
look at other “complete” Idris backends (ex: JavaScript) to see how the
interfacing between Idris
and the target language is implemented

review ''Idris, a General Purpose Dependently Typed Programming Language:
Design and Implementation''
and ''Cross-platform Compilers for Functional Languages'' research papers to
understand the Idris
support for FFIs

review other related literature
study the OCaml and the Idris type systems

\section{Methodology}
\section{Evaluation}
\section{Marking Scheme}
\section{Bob's comments}

% \chapter{Introduction}
% \label{ch:intro}
% \chapter{Technical Introduction}
% \label{ch:techintro}
% \chapter{Method}
% \label{ch:method}
% \chapter{Results}
% \label{ch:results}
% \chapter{Conclusions}
% \label{ch:conc}
% \bibliographystyle{plain}
% \bibliography{thesis}
\end{document}